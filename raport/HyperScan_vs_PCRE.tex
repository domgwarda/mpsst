\documentclass[a4paper,11pt]{article}

\usepackage[T1]{fontenc}        % poprawne łamanie polskich znaków
\usepackage[utf8]{inputenc}     % kodowanie UTF-8
\usepackage[polish]{babel}      % język polski (nowocześniej niż \usepackage{polski})
\usepackage{microtype}

\usepackage{listings}

\setlength{\headheight}{16pt}
\usepackage{enumerate}

\usepackage{amssymb}		% pakiet do symboli
\usepackage{amsmath}		% pakiet do matmy

\usepackage{microtype}          % lepsze łamanie wierszy (bardzo polecane)
\usepackage[a4paper, margin=1.5cm, top=2cm, bottom=1.8cm]{geometry}

\usepackage{enumerate}
\usepackage{wrapfig} % enables wrapfigure environment
\usepackage{subcaption} % for subfigure environment
\usepackage{enumitem}		% punktowanie (a), (b), ...

\usepackage{graphicx}		% wstawianie obrazkow
\usepackage{float}			% wstawianie obrazkow w dowolnym miejscu
% \usepackage{tabstackengine} 	% macierze
\usepackage{fancyhdr}		% ładniejszy wygląd stronki
% \usepackage{color}   		% kolorowanie linków / odnośników
\usepackage{hyperref} %do linków
\hypersetup{
    colorlinks=true, 	% set true if you want colored links
    linktoc=all,     	% set to all if you want both sections and subsections linked
    linkcolor=black,  	% choose some color if you want links to stand out
}
\usepackage{titling}


\usepackage{newclude}
\usepackage{xinttools}
\usepackage{color}

\usepackage[a4paper, margin=1.5cm, top=2cm, bottom=1.8cm]{geometry}


% for + / -
\usepackage{tikz}
\newcommand{\Mypm}{\mathbin{\tikz [x=1.4ex,y=1.4ex,line width=.1ex] \draw (0.0,0) -- (1.0,0) (0.5,0.08) -- (0.5,0.92) (0.0,0.5) -- (1.0,0.5);}}%


\renewcommand\labelitemi{\raisebox{-0.3ex}{\huge\boldmath$\cdot$}} % obniżenie pkt + zmniejszenie
% przeciwko dzieleniu słów np bez- (enter) względny
\hyphenpenalty=10000
\exhyphenpenalty=10000

% \stackMath

% nowe komendy dla wygodniejszego pisania :)
\newcommand{\subtitle}[1]{ \posttitle{ \par\end{center} \begin{center}\large#1\end{center} \vskip0.5em}}
% \newcommand{\floor}[1]{\left\lfloor #1 \right\rfloor}
\newcommand{\ceil}[1]{\left\lceil #1 \right\rceil}

% Wyglad strony:
% \pagestyle{fancy}
\fancyhf{}
\chead{\leftmark}
\cfoot{\thepage}


% \author{}
\title{\textbf{HyperScan vs PCRE}\\
Raport z projektu}
\author{Autorzy: \textbf{}}



% \subtitle{Redukcja}
\date{}

\begin{document}
\thispagestyle{empty}
\maketitle
\mbox{}

\section{Wydajności czasowe}

\subsection{Jak ilość regexów wpływa na czas wykonania się programu?}

\subsubsection{Podział wzorców regex}

W celu analizy wpływu charakteru wyrażeń regularnych na wydajność silników
Hyperscan i PCRE, zastosowane wzorce podzielono na cztery grupy różniące się
złożonością oraz właściwościami obliczeniowymi.

\begin{itemize}
    \item \textbf{Grupa 1 – literały (słowa)}
    Pojedyncze ciągi znaków bez metaznaków regex.
    Brak ograniczeń początku dopasowania, duża liczba potencjalnych trafień.

    \item \textbf{Grupa 2 – regexy literalne z granicami słowa }
    Długie literały ograniczone granicami słowa.
    Mała liczba pozycji startowych i bardzo niska częstość dopasowań.

    \item \textbf{Grupa 3 – regexy strukturalne}
    Wzorce o określonej strukturze (klasy znaków, kwantyfikatory, alternacje).
    Szybkie odrzucanie niedopasowanych fragmentów tekstu.

    \item \textbf{Grupa 4 – regexy z szerokim dopasowaniem}
    Wzorce zawierające wildcardy (\verb|.*|, \verb|.{0,200}|), alternacje i powtórzenia.
    Duża przestrzeń dopasowania i potencjalnie wysoki koszt obliczeniowy.
\end{itemize}

Podział ten umożliwia porównanie zachowania silników dla wzorców o rosnącej
złożoności oraz analizę wpływu typu regexów na skalowanie czasowe.

\newpage

\subsubsection{Wyniki}
\begin{figure}[h!]
\centering

\begin{subfigure}{0.48\textwidth}
    \centering
    \includegraphics[width=\textwidth]{bench_plot.png}
\end{subfigure}
\hfill
\begin{subfigure}{0.48\textwidth}
    \centering
    \includegraphics[width=\textwidth]{bench_plot_regex_easy.png}
\end{subfigure}

\vspace{0.5em}

\begin{subfigure}{0.48\textwidth}
    \centering
    \includegraphics[width=\textwidth]{bench_plot_regex_normal.png}
\end{subfigure}
\hfill
\begin{subfigure}{0.48\textwidth}
    \centering
    \includegraphics[width=\textwidth]{bench_plot_regex_hard.png}
\end{subfigure}


\end{figure}


\subsubsection{Wnioski}

\begin{itemize}
    \item Wraz ze wzrostem liczby wzorców Hyperscan wyraźnie lepiej radzi sobie
    z szukaniem wzorców niż PCRE. Różnica pomiędzy silnikami rośnie wraz
    z liczbą regexów i dla dużych zbiorów osiąga kilka rzędów wielkości.

    \item Najlepsze wyniki dla obu silników uzyskano w przypadku regexów
    strukturalnych. Wzorce o jasno określonej strukturze umożliwiają szybkie
    odrzucanie niedopasowanych fragmentów tekstu, co przekłada się na niski
    i stabilny czas wykonania.

    \item Hyperscan charakteryzuje się bardzo wysoką wydajnością we wszystkich
    analizowanych przypadkach, jednak najlepsze rezultaty uzyskuje dla regexów
    literalnych z granicami słowa oraz dla regexów strukturalnych.

    \item Silnik PCRE zdecydowanie najlepiej radzi sobie właśnie z regexami
    strukturalnymi, osiągając dla nich znacznie lepsze wyniki niż dla pozostałych
    kategorii wzorców. W pozostałych grupach czas działania PCRE rośnie
    zauważalnie szybciej.

    \item Krótkie literały bez ograniczeń początku dopasowania oraz regexy
    z szerokim dopasowaniem i alternacjami stanowią dla Hyperscana większe
    obciążenie, choć nawet w tych przypadkach jego wydajność pozostaje
    znacząco wyższa niż w przypadku PCRE.

\end{itemize}

Uzyskane wyniki wskazują, że im większa jest liczba jednocześnie
    analizowanych wzorców, tym przewaga Hyperscana nad PCRE staje się
    wyraźniejsza, co czyni go szczególnie dobrze przystosowanym do zadań
    typu multi-pattern matching.







\subsection{Jak ilość wątków wpływa na czas wykonania się programu?}
\subsection{Jak ilość regexów vs ilość wątków wpływa na czas wykonania się programu?}

\subsection{Jak ilość małych ($\leq 3$\,MB) plików wpływa na czas wykonywania?}

\begin{figure}[H]
    \centering
    \includegraphics[width=0.8\linewidth]{hs_300rgx_4000files.png}
\end{figure}

Pliki użyte w teście miały nie więcej niż $3$\,MB i zostały wygenerowane poleceniem:
\begin{quote}
\texttt{base64 /dev/urandom | head -c 3MB > ...}
\end{quote}
Do testów wykorzystano 100 wątków, a każdy program miał 3 podejścia do problemu. Wynikiem przedstawionym na wykresie jest średnia arytmetyczna całkowitego czasu wykonania.

\subsubsection{Przykłady regexów użytych w teście}

\begin{itemize}[leftmargin=*, itemsep=2pt]
    \item \lstinline!^[0-9]+(\.[0-9]+)?\s?(l|ml)$!
    \item \lstinline!^Saldo:\s-?[0-9]+,[0-9]{2}\sPLN$!
    \item \lstinline!^Dawka:\s[0-9]+(mg|ml)$!
\end{itemize}

\subsubsection{Wnioski z wykresu}

\begin{itemize}[leftmargin=*, itemsep=2pt]
    \item Do około 100 plików, zarówno HS, jak i PCRE radziły sobie podobnie, z czasem wykonywania od 2 do 200\,ms.
    \item Wraz ze wzrostem liczby plików czas wykonywania PCRE rośnie znacznie szybciej niż HS.
    \item Przyspieszenie PCRE jest zdecydowanie większe niż przyspieszenie HS.
    \item HS lepiej radzi sobie z łączeniem skomplikowanych regexów przy pracy na wielu plikach.
\end{itemize}



\end{document}
